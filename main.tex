\documentclass{article}
\usepackage[utf8]{inputenc}

\title{Using overleaf for writing and integrating with Jupyter and Github}
\author{Arindam Basu}
\date{October 2018}

\usepackage{natbib}
\usepackage{graphicx}

\begin{document}

\maketitle

\section*{Introduction}
Overleaf is an excellent tool and works well for productive work. In this tutorial, I am going to use a low end machine running Arch Linux with pandoc and jupyter notebook and git installed. My goal will be to first create an article (this one) in Overleaf, with initial ideas of plans and data analyses, and then I will use github and git to work with the article by adding analyses to produce an article that will then get pushed to online repositories for publishing. This will demonstrate that a workflow is possible using only plain text integrating latex and markdown and other tools that are not resource heavy but can be used gainfully without the need for using resource intensive and non-free or non-open source tools for data analysis and writing fairly complex texts. 

For this to work, we will use the following tools and processes:

\begin{itemize}
    \item Overleaf as a starter document
    \item Push the document in Overleaf to github
    \item Set up a git repo or clone the git repo on the local machine
    \item In the same directory, open a new jupyter notebook
    \item read data, conduct data analyses
    \item write the document in the jupyter notebook using markdown
    \item convert the notebook *.ipynb file to latex
    \item connect to the github repo using git
    \item read it back in overleaf and change or edit as the final piece

\end{itemize}

With a workflow such as this, it is important to keep some processes separate. First, while overleaf can be a starter and finisher of the document, some processes need to be carefully reviewed such that a latex document created here does not necessarily play well in the jupyter notebook unless using git and pandoc one can faithfully translate the documents, so a caution might be useful. Also, while in jupyter using markdown, it is easy to insert citations, they are best avoided using the markdown syntax of [@citation\_id].

\subsection*{Steps to use Overleaf with Jupyter
notebook}\label{steps-to-use-overleaf-with-jupyter-notebook}

The steps were as follows:

\begin{enumerate}
\def\labelenumi{\arabic{enumi}.}
%\tightlist
\item
  We created an article in Overleaf first
\item
  We then pushed the article to github
\item
  We visited github and copied the address of the article that we'd
  clone
\item
  We then created a new git repository on our machine where we have
  cloned the directory (in our case, we called that ``work'')
\end{enumerate}

Now,

\begin{itemize}
%\tightlist
\item
  We have added a jupyter notebook
\item
  The jupyter notebook will let us add text using markdown documentation
\item
  We will continue to store our bibliography information on the
  references.bib file in bibtex format
\item
  We will add citation information using
  \texttt{\textbackslash{}cite\{citationid\}}
\end{itemize}

\subsection*{Next Steps}\label{next-steps}

We will convert this notebook to latex using two steps: In the first
step, we will convert the notebook to a markdown file, using

\texttt{jupyter\ nbconvert\ test\_section.ipynb} This will convert this
notebook to markdown format. Then we will convert the test\_section.md
to test\_section.tex using pandoc as

\texttt{pandoce\ -f\ markdown\ -t\ latex\ -o\ test\_section.tex\ test\_section.md}

We will then do some more things:

\begin{enumerate}
\def\labelenumi{\arabic{enumi}.}
%\tightlist
\item
  We will create .gitignore file that will contain all files with
  extension \texttt{*.ipynb} as git as problems in maintaining them.
\item
  Optionally, if we do not want to share our markdown files, we will add
  \texttt{*.md} files to the list as well.
\item
  Lastly, and optionally, we can add the new \texttt{test\_section.tex}
  file to the \texttt{main.tex} file using the
  \texttt{\textbackslash{}input\{test\_section\}}
\end{enumerate}

After we have done these three changes, we will then push the files to
the git repo using:

\begin{verbatim}
git pull origin master
git add test_section.tex main.tex
git commit -m "added test_section files"
git push origin master
\end{verbatim}

Then, once we are in Overleaf, we will pull in the changes to Overleaf
and work on Overleaf.

Once you are in Overleaf, you can edit the document and continue to push to github. Each file that you push to github will then be pulled into the computer of your collaborators or your own computer (depending on where you have been working). You can work on a minimal computer that gets connected to the web or a not compatible operating system that can connect to the web and you can use a hosted instance of jupyter notebook where you can work. If that is the case, then you can use the hosted instance to work on the returned or received document and work on it. If you are familiar with latex, then you can directly continue to work on the document in a jupyter notebook environment and push the changes back to overleaf. If you are happy to work with markdown, then you contine to convert the latex file from latex to markdown and work directly on the markdown document and remember to reconvert and push to Overleaf. If you want to continue to work on the test\_section.ipynb, you can do so, but this time when you export to latex, save it with a different name or something like version 2 or somehow that signifies that this file is different from the previous one that you sent. This way, going back and forth, you can combine powerful and weak machines, web and offline, in many different ways to set up a workflow that pulls and pushes changes to the flow of your work. 

The result will be a pdf document (only on Overleaf), but if you want, you can use pandoc at your end to convert the document (or the latex document that results from Overleaf) to word document and continue to publish for those situations where a word document is desired. I recommend that you start with an article template that Overleaf provides rather than open a blank article as this helps you to get things done quickly. 






\begin{figure}[h!]
\centering
\includegraphics[scale=1.7]{universe}
\caption{The Universe}
\label{fig:universe}
\end{figure}

\section*{Conclusion}
``I always thought something was fundamentally wrong with the universe'' \citep{adams1995hitchhiker}

\bibliographystyle{plain}
\bibliography{references}
\end{document}
